\documentclass[a4paper,12pt]{article}

\usepackage{graphicx}
\usepackage{amsmath,amsfonts,amsthm,amstext,amssymb}
\usepackage[brazil]{babel}

\usepackage{color}      
% Mudei aqui pois o Overleaf usa Linux para gerar, então precisa ser em utf8
\usepackage[utf8]{inputenc}
% Novos pacotes para rodar bibliografia. Style = numeric-comp é para ser numerado e 
% sorting = nyt usa nome, ano e título para ordenar
% No link http://tug.ctan.org/info/biblatex-cheatsheet/biblatex-cheatsheet.pdf tem tudo sobre biblatex
\usepackage{csquotes}
\usepackage[backend = bibtex,style=numeric-comp,sorting=nyt,date=year,isbn=false,maxnames=10]{biblatex}
\addbibresource{bibliografianova.bib}
\usepackage[a4paper,top=2cm,bottom=2cm,left=1.5cm,right=1.5cm]{geometry}
\usepackage{setspace} \onehalfspacing
\usepackage{lastpage}

\newcommand{\R}{\mathbb{R}}
\newcommand{\N}{\mathbb{N}}


\newcommand{\tb}{\mathbf{t}}
\newcommand{\RRR}{\mathbf{R}}
\newcommand{\bsrho}{\boldsymbol{\rho}}
\newcommand{\uu}{\mathbf{u}}
\newcommand{\fm}{\mathbf{f}}
\newcommand{\ds}{\displaystyle}

\usepackage{multirow}
\usepackage[normalem]{ulem}
\useunder{\uline}{\ul}{}

\title{\large\textbf{Resumo para o Congresso Acadêmico UNIFESP 2020} \\
%\vspace{.4cm} 
Métodos de Otimização aplicados a redes neurais para \\ detecção de anomalias em transações com cartão de crédito}

\author{\large\textbf{Aluno:} Frederico José Ribeiro Pelogia\thanks{Graduando do Bacharelado em Ciência e Tecnologia,  Universidade Federal de São Paulo,
Campus  São José dos Campos. Email: fredpelogia@outlook.com}
\and  
\textbf{Orientador:} Luís Felipe Bueno\thanks{Departamento de Ciência e Tecnologia, Instituto de Ciência e Tecnologia,  Universidade Federal de
 São Paulo, Campus  São José dos Campos. Email: lfelipebueno@gmail.com}}

\date{Abril de 2020}



\begin{document}
	

	
\maketitle

\abstract A detecção de transações fraudulentas de cartão de crédito tem expressiva importância nos tempos atuais, tendo em vista o grande volume de operações desse tipo. A utilização de técnicas de aprendizado de máquina tem se mostrado eficiente para detecção de anomalias\cite{Chandola:2009} e, por esse motivo, diversas instituições financeiras possuem setores especializados nesse tipo de algoritmo.
Este trabalho apresentará os fundamentos de redes neurais, assim como alguns dos métodos de otimização, em sua maioria estocásticos,  mais utilizados atualmente para fazer o treinamento dessas. Também serão apresentados os resultados preliminares da aplicação dos algoritmos implementados ao problema de detecção de fraudes, utilizando a base de dados “Credit Card Fraud Detection”, da plataforma Kaggle. Um dos principais pontos para o restante do projeto será desenvolver uma versão estocástica do algoritmo de primeira ordem proposto em \cite{bmLS}, que obteve bons resultados de complexidade para problemas de quadrados mínimos.\\



\noindent {\bf Palavras Chave:} {Otimização, Métodos Estocásticos, Redes Neurais, Deep Learning, Detecção de Anomalias.}


% Mudei a forma da bibliografia
%\bibliographystyle{acm}
% \bibliographystyle{plain}
% \bibliography{bibliografianova}
\printbibliography
\end{document}



